\documentclass[10pt, a4paper]{article}
% especifico m?rgenes manualmente
\usepackage[paper=a4paper, left=1.5cm, right=1.5cm, bottom=1.5cm, top=3.5cm]{geometry}
% codificaci?n ISO-8859-1
%\usepackage[utf8]{inputenc}
\usepackage[utf8]{inputenc}
% separaci?n sil?bica en castellano
\usepackage[spanish]{babel}
% paquetes y caratula de algo2
\usepackage{aed2-symb,aed2-itef,aed2-tad,caratula}

\begin{document}

% Estos comandos deben ir antes del \maketitle
\materia{Algoritmos y Estructuras de Datos II} % obligatorio
\submateria{Segundo Cuatrimestre de 2015} % opcional
\titulo{Trabajo Pr\'actico 1} % obligatorio
\subtitulo{Especificaci\'on Alta Seguridad nos cuida} % opcional
\grupo{Grupo 18} % opcional 


\integrante{Fernando Frassia}{340/13}{ferfrassia@gmail.com} % obligatorio 
\integrante{Sebastian Matias Giambastiani}{916/12}{sebastian.giambastiani@hotmail.com}
\integrante{Mat\'ias Millass\'on}{131/13}{matiasmillasson@gmail.com}
\integrante{Rafael Ceniceros}{325/14}{rafaceniceros@gmail.com}



\maketitle

% compilar 2 veces para actualizar las referencias
\tableofcontents

\pagebreak
\newpage

\section{Aclaraciones}
Tomamos como decisiones de especificaci\'on lo siguiente:
\begin{description}
    \item{$\bullet$ un hippie se convierte en estudiante $\Leftrightarrow$ est\'a inmovilizado $\wedge$ lo rodean todos estudiantes en todos sus 'movimientos posibles', y ning\'un obstaculo.}
    \item{$\bullet$ un hippie es capturado $\Leftrightarrow$ est\'a inmovilizado $\wedge$ lo rodea al menos 1 polic\'ia.} 
    \item{$\bullet$ un estudiante se convierte en hippie $\Leftrightarrow$ lo rodean al menos 2 hippies (no importa su capacidad de moverse).}
    \item{$\bullet$ la grilla, de tamanio n x m, va de 1 a n y de 1 a m.}
\end{description}
Tambi\'en decidimos tomar a los Estudiantes, Hippies y Polic\'ias como \tadNombre{Nat}, reconocemos que se puede sobreespecificar de esta forma, porque ser\'ia m\'as prolijo separar a los Estudiantes y Hippies por un lado y a los Polic\'ias por otro, pero consultamos y nos dijeron que se admit\'ia este tipo de dise�o. Lo mismo sucede con Rol como \tadNombre{String} en lugar de un tipo enumerado.\\

Por \'ultimo, las Coodenadas son un renombre de \tadNombre{Tupla(Nat, Nat)}, donde la componente $\pi_1$ es x, y la componente $\pi_2$ es y.
\newpage
 

%RenombresBegin
\section{Renombres}

\tadNombre{ID} es \tadNombre{Nat}

\tadNombre{Chabon} es \tadNombre{Nat}

\tadNombre{Rol} es \tadNombre{String}

\tadNombre{Coordenada} es \tadNombre{Tupla(Nat, Nat)}

%\tadNombre{Paquete} es \tadNombre{Tupla(Nat, Nat), el $\pi$$_1$ de la tupla es la prioridad del paquete, y el $\pi$$_2$ es su identificador personal}

%\tadNombre{} es \tadNombre{}
%RenombresEnd

%TADSBegin

%SecuExtendidoBegin



%\end{tad}
%SecuExtendidoEnd
\newpage

%RedBegin
\section{TAD \tadNombre{Grilla}}

\begin{tad}{\tadNombre{Grilla}}
\tadIgualdadObservacional{g1}{g2}{Grilla}{tam($g_1$) $\igobs$ tam($g_2$) $\yluego$ ($\forall$ cor: Coordenada) enRango(cor, $g_1$) $\impluego$ libre?(cor, $g_1$) $\igobs$ libre?(cor, $g_2$)}
\tadGeneros{Grilla}


\tadExporta{Grila, generadores, observadores, otras operaciones}
%\tadUsa{\tadNombre{Bool}, \tadNombre{Conj($\alpha$)}, \tadNombre{Compu}, \tadNombre{Interfaz}, \tadNombre{Secu($\alpha$)}}

\tadObservadores
\tadAlinearFunciones{libre?}{Coordenada/cor, Grilla/g}
\tadOperacion{libre?}{Coordenada/cor, Grilla/g}{Bool}{enRango(cor, g)}
\tadOperacion{tam}{Grilla}{Coordenada}{}


\tadGeneradores
\tadAlinearFunciones{obstaculizar}{Coordenada/cor, Grilla/g}
\tadOperacion{crear}{Coordenada/cor}{Grilla}{}
\tadOperacion{obstaculizar}{Coordenada/cor, Grilla/g}{Grilla}{enRango(cor, g) $\yluego$ libre?(cor, g)}

\tadOtrasOperaciones
\tadAlinearFunciones{posLibresAux}{Coordenada/cor1, Coordenda/cor2, Grilla/g}

\tadOperacion{enRango}{Coordenada/cor, Grilla/g}{Bool}{}

\tadOperacion{posTotales}{Grilla}{Conj(Coordenada)}{}

\tadOperacion{posTotalesAux}{Coordenada/cor1, Coordenda/cor2, Grilla/g}{Conj(Coordenada)}{}

\tadOperacion{posLibres}{Grilla}{Conj(Coordenada)}{}

\tadOperacion{posLibresAux}{Coordenada/cor1, Coordenada/cor2, Grilla/g}{Conj(Coordenada)}{}

\tadOperacion{posOcupadas}{Grilla}{Conj(Coordenada)}{}

  

\tadAxiomas[\paratodo{Coordenada}{cor, cor1, cor2}, \paratodo{Grilla}{g}]
\tadAlinearAxiomas{libre?($cor_1$ obstaculizar($cor_2$, g))}

\tadAxioma{libre?($cor_1$, crear($cor_2$))}{True}

\tadAxioma{libre?($cor_1$ obstaculizar($cor_2$, g))}{\IF $cor_1$ = $cor_2$ THEN False ELSE libre?($cor_1$ g) FI}

\tadAxioma{tam(crear(cor))}{cor}

\tadAxioma{tam(obstaculizar(cor, g))}{tam(g)}

\tadAxioma{enRango(cor, g)}{cor.x $\leq$ tam(g).x $\wedge$ cor.y $\leq$ tam(g).y}

\tadAxioma{posTotales(g)}{posTotalesAux((1, 1), tam(g), g)}

\tadAxioma{posTotalesAux($cor_1$, $cor_2$, g)}{\IF $cor_1$.y $\leq$ $cor_2$.y THEN 
							            	{\IF $cor_1$.x $\leq$ $cor_2$.x THEN 
								            	Ag($cor_1$, posTotalesAux(($cor_1$.x+1, $cor_1$.y), $cor_2$, g)) 
							            	ELSE 
								            	posTotalesAux((1, $cor_1$.y+1), $cor_2$, g) FI} 
						            	ELSE 
							            	$\emptyset$ FI}

\tadAxioma{posLibres(g)}{posLibresAux((1, 1), tam(g), g)}

\tadAxioma{posLibresAux($cor_1$, $cor_2$, g)}{\IF $cor_1$.y $\leq$ $cor_2$.y THEN 
								            	{\IF $cor_1$.x $\leq$ $cor_2$.x THEN 
									            	{\IF libre?($cor_1$, g) THEN 
									            		Ag($cor_1$, posLibresAux(($cor_1$.x+1, $cor_1$.y), $cor_2$, g)) 
									            	ELSE 
									            		posLibresAux(($cor_1$.x+1, $cor_1$.y), $cor_2$, g) FI} 
								            	ELSE 
									            	posLibresAux((1, $cor_1$.y+1), $cor_2$, g) FI} 
							            	ELSE 
								            	$\emptyset$ 
									FI}
							            	
\tadAxioma{posOcupadas(g)}{posTotales(g) - posLibres(g)}



\end{tad}
%RedEnd

\newpage

%DCNetBegin
\section{TAD \tadNombre{Sistema}}

\begin{tad}{\tadNombre{Sistema}}
\tadIgualdadObservacional{s1}{s2}{Sistema}{laGrilla($s_1$) $\igobs$ laGrila($s_2$) $\yluego$ \\ 
									($\forall$ c: Chabon) c $\in$ gente($s_1$) $\igobs$ c $\in$ gente($s_2$) $\yluego$ \\
									($\forall$ c: Chabon) c $\in$ gente($s_1$) $\impluego$ \\
									pos(c, $s_1$) $\igobs$ pos(c, $s_2$) $\wedge$ \\
									rol(c, $s_1$) $\igobs$ rol(c, $s_2$) $\yluego$ \\
									(rol(c, $s_1$) = polic\'ia $\impluego$ infraccionesDe(c, $s_1$) $\igobs$ infraccionesDe(c, $s_2$) $\wedge$ capturasDe(c, $s_1$) $\igobs$ capturasDe(c, $s_2$))
									 }
\tadGeneros{Sistema}

\tadExporta{Sistema, generadores, observadores}
%\tadUsa{\tadNombre{Red}, \tadNombre{Secu($\alpha$)}, \tadNombre{Conjunto($\alpha$)}, \tadNombre{Nat}, \tadNombre{Bool}, \tadNombre{Paquete}, \tadNombre{Tupla}}

\tadObservadores
\tadAlinearFunciones{infraccionesDe}{Chabon/c, Sistema/s}
\tadOperacion{laGrilla}{Sistema}{Grilla}{} 
\tadOperacion{gente}{Sistema}{Conj(Chabon)}{} 
\tadOperacion{rol}{Chabon/c, Sistema/s}{Rol}{c $\in$ gente(s)} 
\tadOperacion{pos}{Chabon/c, Sistema/s}{Coordenada}{c $\in$ gente(s)} 
\tadOperacion{infraccionesDe}{Chabon/c, Sistema/s}{Nat}{c $\in$ gente(s) $\yluego$ rol(c, s) = polic\'ia} 
\tadOperacion{capturasDe}{Chabon/c, Sistema/s}{Nat}{c $\in$ gente(s) $\yluego$ rol(c, s) = polic\'ia} 


\tadGeneradores
\tadAlinearFunciones{moverNoEst}{Chabon/c, Bool/b, Nat/i, Nat/j, Sistema/s}
\tadOperacion{rastrillar}{Conj(Chabon)/cc, Grilla/g}{Sistema}{$\#$cc $\leq$ $\#$posLibres(g)}
\tadOperacion{entraNoPoli}{Chabon/c, Rol/r, Coordenada/cor, Sistema/s}{Sistema}{c $\notin$ gente(s) $\wedge$ cor $\in$ posLibresSist(s) $\wedge$ r $\in$ \{estudiante, hippie\} $\wedge$ (cor.$y$ = 1 $\vee$ cor.$y$ = tam(laGrilla(s)).$y$)}
\tadOperacion{moverEst}{Chabon/c, Coordenada/cor, Sistema/s}{Sistema}{c $\in$ gente(s) $\yluego$ rol(c, s) = estudiante $\wedge$ cor $\in$ adyacentesLibres(pos(c, s), s)}
\tadOperacion{moverNoEst}{Chabon/c, Sistema/s}{Sistema}{c $\in$ gente(s) $\yluego$ rol(c, s) $\neq$ estudiante $\wedge$ \#adyacentesLibres(pos(c), s) $>$ 0 $\wedge$ (rol(c, s) = polic\'ia $\impluego$ infraccionesDe(c, s) $\leq$ 3)}

%%%%%%%%%%%%%%%%%%%%%%%
%			INICIO				   %
%          OTRAS OPERACIONES                 %
%								   %
%								   %	
%%%%%%%%%%%%%%%%%%%%%%%


\tadOtrasOperaciones
\tadAlinearFunciones{posOcupadasPorGenteAux}{Chabon/c, Nat/i, Nat/j, Sistema/s} 

\tadOperacion{IDpolic\'ia}{Chabon/c, Sistema/s}{ID}{c $\in$ gente(s)}

\tadOperacion{cantHippies}{Sistema}{Nat}{}

\tadOperacion{cantEstudiantes}{Sistema}{Nat}{}

\tadOperacion{elMasVigilante}{Sistema}{Chabon}{$\neg$$\emptyset$?losPolic\'ias(s)}

\tadOperacion{adyacentesConGente}{Coordenada/cor, Sistema/s}{Conj(Coordenada)}{enRango(cor, laGrilla(s))} 

\tadOperacion{adyacentesConHippies}{Coordenada/cor, Sistema/s}{Conj(Coordenada)}{enRango(cor, laGrilla(s))} 

\tadOperacion{adyacentesLibres}{Coordenada/cor, Sistema/s}{Conj(Coordenada)}{enRango(cor, laGrilla(s))} 

\tadOperacion{adyacentesOcupadas}{Coordenada/cor, Sistema/s}{Conj(Coordenada)}{enRango(cor, laGrilla(s))} 

\tadOperacion{adyacentesTotales}{Coordenada/cor, Sistema/s}{Conj(Coordenada)}{enRango(cor, laGrilla(s))} 

\tadOperacion{adyacentesEstudiantes}{Coordenada/cor, Sistema/s}{Conj(Coordenada)}{enRango(cor, laGrilla(s))}

\tadOperacion{adyacentesHippies}{Coordenada/cor, Sistema/s}{Conj(Coordenada)}{enRango(cor, laGrilla(s))}

\tadOperacion{adyacentespolic\'ias}{Coordenada/cor, Sistema/s}{Conj(Coordenada)}{enRango(cor, laGrilla(s))}

\tadOperacion{cantEnemigosCerca}{Conj(Chabon)/cc, Rol/r, Sistema/s}{Nat}{cc $\subseteq$ gente(s) $\wedge$ r $\in$ \{estudiante, hippie, polic\'ia\}} 

\tadOperacion{chabonesConRol}{Conj(Chabon)/cc, Rol/r, Sistema/s}{Conj(Chabon)}{cc $\subseteq$ gente(s) $\wedge$ r $\in$ \{estudiante, hippie, polic\'ia\}}

\tadOperacion{chabonEnPosicion}{Coordenada/cor, Sistema/s}{Chabon}{enRango(cor, laGrilla(s)) $\wedge$ cor $\in$ posOcupadasPorGente(s)} 

\tadOperacion{chabonesEnPosiciones}{Conj(Coordenada)/ccor, Sistema/s}{Conj(Chabon)}{($\forall$ cor $\in$ ccor) enRango(cor, laGrilla(s)) $\wedge$ cor $\in$ posOcupadasPorGente(s)} 

\tadOperacion{corALaQueVa}{Chabon/c, Sistema/s}{Coordenada}{c $\in$ gente(s) $\wedge$ \#adyacentesLibres(pos(c), s) $>$ 0}

\tadOperacion{conjLosTargetsMasCercanos}{Chabon/c, Conj(Coordenada)/ccor, Nat/n, Sistema/s}{Conj(Coordenada)}{c $\in$ gente(s) $\wedge$ ($\forall$ cor $\in$ ccor) enRango(cor, laGrilla(s))}

\tadOperacion{dameMenorPlaca}{Conj(Chabon)/cc, Chabon}{Chabon}{$\neg$$\emptyset$?(cc)}

\tadOperacion{diccChabonRol}{Conj(Coordenada)/ccor, Sistema/s}{Diccionario(Coordenada, Rol)}{($\forall$ cor $\in$ ccor) enRango(cor, laGrilla(s))}

\tadOperacion{distancia}{Chabon/c1, Chabon/c2, Sistema/s}{Nat}{$\{$$c_1$, $c_2$$\}$ $\subseteq$ gente(s)} 

\tadOperacion{distUnTargetMasCercano}{Chabon/c, Conj(Coordenada)/ccor, Nat/n, Sistema/s}{Nat}{c $\in$ gente(s) $\wedge$ ($\forall$ cor $\in$ ccor) enRango(cor, laGrilla(s))}

\tadOperacion{loRodeanBien?}{Conj(Chabon)/cc, Nat/n, Rol/r, Sistema/s}{Bool}{cc $\subseteq$ gente(s) $\wedge$ r $\in$ \{estudiante, hippie, polic\'ia\}} 

\tadOperacion{losEstudiantes}{Sistema}{Conj(Chabon)}{} 

\tadOperacion{losHippies}{Sistema}{Conj(Chabon)}{} 

\tadOperacion{lospolic\'ias}{Sistema}{Conj(Chabon)}{} 

\tadOperacion{posDeCaptura?}{Coordenada/cor, Sistema/s}{Bool}{enRango(cor, laGrilla(s))} 

\tadOperacion{posDeCapturas}{Conj(Coordenada)/ccor, Sistema/s}{Conj(Chabon)}{($\forall$ cor $\in$ ccor) enRango(cor, laGrilla(s))} 

\tadOperacion{posEstudiantes}{Sistema}{Conj(Coordenada)}{} 

\tadOperacion{posHippies}{Sistema}{Conj(Coordenada)}{} 

\tadOperacion{posPolic\'ias}{Sistema}{Conj(Coordenada)}{} 

\tadOperacion{posQueSeConvierte?}{Coordenada/cor, Rol/r, Sistema/s}{Bool}{enRango(cor, laGrilla(s)) $\wedge$ r $\in$ \{estudiante, hippie\}} 

\tadOperacion{posQueSeConvierten}{Conj(Coordenada)/ccor, Diccionario(Coordenada, Rol)/d, Sistema/s}{Conj(Coordenada)}{($\forall$ cor $\in$ ccor $\cup$ claves(d)) enRango(cor, laGrilla(s)) $\yluego$ ($\forall$ cor $\in$ claves(d)) obtener(cor, d) $\in$ \{estudiante, hippie\}}

\tadOperacion{posInmovil?}{Coordenada/cor, Sistema/s}{Bool}{enRango(cor, laGrilla(s))} 

\tadOperacion{posInmoviles}{Conj(Coordenada)/ccor, Sistema/s}{Conj(Coordenada)}{($\forall$ cor $\in$ ccor) enRango(cor, laGrilla(s))}

\tadOperacion{posLibresSist}{Sistema}{Conj(Coordenada)}{} 

\tadOperacion{posOcupadasSist}{Sistema}{Conj(Coordenada)}{} 

\tadOperacion{posOcupadasPorGente}{Sistema}{Conj(Coordenada)}{} 

\tadOperacion{recordCapturas}{Sistema}{Nat}{} 

\tadOperacion{losMasVigilantes}{Conj(Chabon)/cc, Nat/n, Sistema/s}{Conj(Chabon)}{cc $\subseteq$ gente(s)}

\tadOperacion{posInicialAux}{Chabon/c, Conj(Chabon)/cc, Conj(Coordenada)/ccor}{Coordenada}{Ag(c, cc) $\subseteq$ gente(s) $\wedge$ ($\forall$ cor $\in$ ccor) enRango(cor, laGrilla(s))}

\tadOperacion{posOcupadasPorGenteAux}{Conj(Chabon)/cc, Sistema/s}{Conj(Coordenada)}{cc $\subseteq$ gente(s)} 

\tadOperacion{recordCapturasAux}{Conj(Chabon)/cc, Nat/n, Sistema/s}{Nat}{cc $\subseteq$ gente(s)}

\tadOperacion{chabonEnPosicionAux}{Conj(Chabon)/cc, Coordenada/cor, Sistema/s}{Chabon}{ cc $\subseteq$ gente(s) $\wedge$ enRango(cor, laGrilla(s)) $\wedge$ cor $\in$ posOcupadasPorGente(s)}

\tadOperacion{distanciaAux}{Coordenada/c1, Coordenada/c2}{Nat}{} 


%%%%%%%%%%%%%%%%%%%%%%%
%			INICIO				   %
% AXIOMAS DE OBSERVADORES             %
%								   %
%								   %	
%%%%%%%%%%%%%%%%%%%%%%%

\tadAxiomas[\paratodo{Conj(Chabon)}{cc}, \paratodo{Grilla}{g}, \paratodo{Rol}{r}, \paratodo{Chabon}{c, c1, c2}, \paratodo{Coordenada}{cor}, \paratodo{Sistema}{s}, \paratodo{Nat}{n}, \paratodo{Conj(Coordenada)}{ccor}, \paratodo{Diccionario(Coordenada, Rol)}{d}]
\tadAlinearAxiomas{infraccionesDe($c_1$, entraNoPoli($c_2$, r, cor, s)}

\tadAxioma{laGrilla(rastrillar(cc, g))}{g}
\tadAxioma{laGrilla(entraNoPoli(c, r, cord, s))}{laGrilla(s)}
\tadAxioma{laGrilla(moverEst(c, cord, s))}{laGrilla(s)}
\tadAxioma{laGrilla(moverNoEst(c, s))}{laGrilla(s)}

\tadAxioma{gente(rastrillar(cc, g))}{cc}
\tadAxioma{gente(entraNoPoli(c, r, cor, s))}{\IF (r = hippie $\wedge$ posDeCaptura?(cor, s)) $\vee$ (r = estudiante $\wedge$ posDeConvertirse?(cor, estudiante, s) $\wedge$ posDeCaptura?(cor, s)) THEN
									gente(s)
								ELSE
									Ag(c, gente(s) -  chabonesEnPosiciones(posDeCapturas(posQueSeConvierten(adyacentesEstudiantes \\ 
									(cor, s), diccChabonRol(adyacentesEstudiantes(cor, s), s), entraNoPoli(c, r, cor, s)) $\cup$ adyacentesHippies(cor, s), entraNoPoli(c, r, cor, s)), entraNoPoli(c, r, cor, s)))
								FI}

\tadAxioma{gente(moverEst(c, cor, s))}{\IF cor.y = 0 $\vee$ cor.y = tam(laGrilla(s)).$y$ + 1 $\vee$ posDeConvertirse?(cor, estudiante, s) $\wedge$ posDeCaptura?(cor, s) THEN 
			                                        		gente(s) - $\{$c$\}$ 
			                                        ELSE
									gente(s) -  chabonesEnPosiciones(posDeCapturas(posQueSeConvierten(adyacentesEstudiantes \\
									(cor, s), diccChabonRol(adyacentesEstudiantes(cor, s), s), moverEst(c, cor, s)) $\cup$ adyacentesHippies(cor, s), moverEst(c, cor, s)), moverEst(c, cor, s))
			                                        FI}

\tadAxioma{gente(moverNoEst(c, s))}{\IF rol(c, s) = hippie THEN
			                                            {\IF posDeCaptura?(corALaQueVa(c, s), moverNoEst(c, s)) THEN
			                                                gente(s) - c
			                                            ELSE
			                                                gente(s) - chabonesEnPosiciones(posDeCapturas(posQueSeConvierten\\ (adyacentesEstudiantes(cor, s), diccChabonRol(adyacentesEstudiantes(cor, s), s), entraNoPoli(c, r, cor, s)) $\cup$ adyacentesHippies(cor, s), entraNoPoli(c, r, cor, s)), entraNoPoli(c, r, cor, s))
			                                            FI}
			                                    ELSE
			                                         gente(s) - chabonesEnPosiciones(posDeCapturas(adyacentesHippies(pos(c, moverNoEst(c, s)), s), moverNoEst(c, s)), moverNoEst(c, s))
			                                    FI}

\tadAxioma{rol(c, rastrillar(cc, g))}{polic\'ia}
\tadAxioma{rol($c_1$, entraNoPoli($c_2$, r, cor, s))}
{\IF rol($c_1$, s) $\neq$ polic\'ia THEN
    {\IF $c_1$ = $c_2$ THEN
        {\IF posQueSeConvierte?(cor, r, s)THEN
            {\IF r = estudiante THEN
                hippie
            ELSE
                estudiante
            FI}
        ELSE
            r
        FI}
    ELSE
        {\IF posQueSeConvierte?(pos($c_1$, s), rol($c_1$, s), entraNoPoli($c_2$, r, cor, s))THEN
            {\IF rol($c_1$, s) = estudiante THEN
                hippie
            ELSE
                estudiante
            FI}
        ELSE
            rol($c_1$, s)
        FI}
    FI}
ELSE
    rol($c_1$, s)
FI}

\tadAxioma{rol($c_1$, moverEst($c_2$, cor, s))}
{\IF rol($c_1$, s) $\neq$ polic\'ia THEN
    {\IF posQueSeConvierte?(pos($c_1$, moverEst($c_2$, cor, s)), rol($c_1$, s), moverEst($c_2$, cor, s))THEN
            {\IF rol($c_1$, s) = estudiante THEN
                hippie
            ELSE
                estudiante
            FI}
        ELSE
            rol($c_1$, s)
        FI}
ELSE
    rol($c_1$, s)
FI}

\tadAxioma{rol($c_1$, moverNoEst($c_2$, s))}
{\IF rol($c_1$, s) $\neq$ polic\'ia THEN
    {\IF posQueSeConvierte?(pos($c_1$, moverNoEst($c_2$, s)), rol($c_1$, s), moverNoEst($c_2$, s))THEN
            {\IF rol($c_1$, s) = estudiante THEN
                hippie
            ELSE
                estudiante
            FI}
        ELSE
            rol($c_1$, s)
        FI}
ELSE
    rol($c_1$, s)
FI}

\tadAxioma{pos(c, rastrillar(cc, g))} {posInicialAux(c, cc, posLibres(g))}						

\tadAxioma{pos($c_1$, entraNoPoli($c_2$, r, cor, s))} {
	\IF $c_1$ = $c_2$ THEN
		cor
	ELSE
		pos($c_1$, s)
	FI
}

\tadAxioma{pos($c_1$, moverEst($c_2$, cor, s))} {
	\IF $c_1$ = $c_2$ THEN
		cor
	ELSE
		pos($c_1$, s)
	FI
}

\tadAxioma{pos($c_1$, moverNoEst($c_2$, s))} {
	\IF $c_1$ = $c_2$ THEN
		corALaQueVa($c_1$, s)
	ELSE
		pos($c_1$, s)
	FI
}

\tadAxioma{infraccionesDe(c, rastrillar(cc,g))}{0}
\tadAxioma{infraccionesDe($c_1$, entraNoPoli($c_2$, r, cor, s)}{\#posInmoviles(adyacentesEstudiantes(pos($c_1$, s), entraNoPoli($c_2$, r, cor, s)), entraNoPoli($c_2$, r, cor, s)) + infraccionesDe($c_1$, s)}
\tadAxioma{infraccionesDe($c_1$, moverEst($c_2$, cor, s)}{\#posInmoviles(adyacentesEstudiantes(pos($c_1$, s), moverEst($c_2$, cor, s), moverEst($c_2$, cor, s)) + infraccionesDe($c_1$, s)}
\tadAxioma{infraccionesDe($c_1$, moverNoEst($c_2$, s)}{\#posInmoviles(adyacentesEstudiantes(pos($c_1$, moverNoEst($c_2$, s)), moverNoEst($c_2$, s)), moverNoEst($c_2$, s)) + infraccionesDe($c_1$, s)}

\tadAxioma{capturasDe(c, rastrillar(cc, g))}{0}
\tadAxioma{capturasDe($c_1$, entraNoPoli($c_2$, r, cor, s))}{\#posInmoviles(adyacentesHippies(pos($c_1$, s), entraNoPoli($c_2$, r, cor, s)), entraNoPoli($c_2$, r, cor, s)) + infraccionesDe($c_1$, s)}
\tadAxioma{capturasDe($c_1$, moverEst($c_2$, cor, s))}{\#posInmoviles(adyacentesHippies(pos($c_1$, s), entraNoPoli($c_2$, r, cor, s)), entraNoPoli($c_2$, r, cor, s)) + infraccionesDe($c_1$, s)}
\tadAxioma{capturasDe($c_1$, moverNoEst($c_2$, s))}{\#posInmoviles(adyacentesHippies(pos($c_1$, moverNoEst($c_2$, s)), entraNoPoli($c_2$, r, cor, s)), entraNoPoli($c_2$, r, cor, s)) + infraccionesDe($c_1$, s)}

%%%%%%%%%%%%%%%%%%%%%%%
%			INICIO				   %
% AXIOMAS DE OTRAS OPERACIONES   %
%								   %
%								   %	
%%%%%%%%%%%%%%%%%%%%%%%

\tadAxioma{IDpolic\'ia(c, s)}{c}

\tadAxioma{cantHippies(s)}{\#posHippies(s)}

\tadAxioma{cantEstudiantes(s)}{\#posEstudiantes(s)}

\tadAxioma{elMasVigilante(s)}{dameMenorPlaca(losMasVigilantes(losPolic\'ias(s), recordCapturas(s), s), dameUno(losMasVigilantes(losPolic\'ias(s), recordCapturas(s), s)))}

\tadAxioma{adyacentesConGente(cor, s)}{posOcupadasPorGente(s) $\cap$  adyacentesTotales(cor, s)}

\tadAxioma{adyacentesConHippies(c, s)}{adyacentesConGente(c, s) $\cap$ posHippies(s)}

\tadAxioma{adyacentesLibres(cor, s)}{posLibresSist(s) $\cap$ adyacentesTotales(cor, s)}

\tadAxioma{adyacentesOcupadas(cor, s)}{posOcupadasSist(s)  $\cap$  adyacentesTotales(cor, s)}
      
\tadAxioma{adyacentesTotales(cor, s)}{\{(cor.$x$+1, cor.$y$), \\(cor.$x$-1, cor.$y$), \\(cor.$x$, cor.$y$+1), \\(cor.$x$, cor.$y$-1)\} $\cap$ posTotales(laGrilla(s))}

\tadAxioma{adyacentesEstudiantes(cor, s)}{adyacentesConGente(cor, s) $\cap$ posEstudiantes(s)}

\tadAxioma{adyacentesHippies(cor, s)}{adyacentesConGente(cor, s) $\cap$ posHippies(s)}

\tadAxioma{adyacentespolic\'ias(cor, s)}{adyacentesConGente(cor, s) $\cap$ pospolic\'ias(s)}

\tadAxioma{cantEnemigosCerca(cc, r, s)}{\IF $\emptyset$?(cc) THEN 
									0 
								ELSE 
									{\IF rol(dameUno(cc), s) = r THEN
										1 + cantEnemigosCerca(sinUno(cc), r, s)
									ELSE
										cantEnemigosCerca(sinUno(cc), r, s)
									FI}
								FI}  
                            
\tadAxioma{chabonesConRol(cc, r, s)}{\IF $\emptyset$?(cc) THEN
                                        $\emptyset$
                                    ELSE
                                        {\IF rol(dameUno(cc), s) = r THEN
                                            Ag(dameUno(cc), chabonesConRol(sinUno(cc), r, s))
                                        ELSE
                                            chabonesConRol(sinUno(cc), r, s)
                                        FI}
                                    FI}                            
                            
\tadAxioma{chabonEnPosicion(cor, s)}{chabonEnPosicionAux(gente(s), cor, s)}

\tadAxioma{chabonesEnPosiciones(ccor, s)}{\IF $\emptyset$?(ccor) THEN
                                            $\emptyset$
                                        ELSE
                                            Ag(chabonEnPosicion(dameUno(ccor), s), chabonesEnPosiciones(sinUno(ccor), s))
                                        FI}
                                        
\tadAxioma{corALaQueVa(c, s)}{\IF (rol(c, s) = polic\'ia $\wedge$ \#posHippies(s) = 0) $\vee$ (rol(c, s) = hippie $\wedge$ \#posEstudiantes(s) = 0) THEN
							{\IF pos(c, s).y $<$ tam(laGrilla(s)).y / 2 THEN
								{\IF (pos(c, s).x, pos(c, s).y - 1) $\in$ posLibresSist(s) THEN
									(pos(c, s).x, pos(c, s).y - 1)
								ELSE 
									dameUno(adyacentesLibres(pos(c, s)))
								FI}
							ELSE
								{\IF (pos(c, s).x, pos(c, s).y + 1) $\in$ posLibresSist(s) THEN
									(pos(c, s).x, pos(c, s).y + 1)
								ELSE
									dameUno(adyacentesLibres(pos(c, s)))
								FI}
							FI}
						ELSE
							{\IF rol(c, s) = hippie THEN
								dameUno(conjLosTargetsMasCercanos(c, posEstudiantes(s), distUnTargetMasCercano(c, posEstudiantes(s), max(tam(laGrilla(s)).x, tam(laGrilla(s)).y), s), s))
							ELSE
								dameUno(conjLosTargetsMasCercanos(c, posEstudiantes(s), distUnTargetMasCercano(c, posHippies(s), max(tam(laGrilla(s)).x, tam(laGrilla(s)).y), s), s))
							FI}	
						FI}                                        
                                        
\tadAxioma{dameMenorPlaca(cc, c)}{\IF $\emptyset$?(cc) THEN
                                        c
                                    ELSE
                                        {\IF dameUno(cc) $<$ c THEN
                                            dameMenorPlaca(sinUno(cc), dameUno(cc))
                                        ELSE
                                            dameMenorPlaca(sinUno(cc), c)
                                        FI}
                                    FI}
                                        
\tadAxioma{diccChabonRol(ccor, s)}{\IF $\emptyset$?(ccor) THEN
								vacio
							ELSE
								definir(dameUno(ccor), rol(chabonEnPosicion(dameUno\\ 
								(ccor), s), s), diccChabonRol(sinUno(ccor, s)))
							FI}                                        
                                        
\tadAxioma{distancia(c1, c2, s)}{distanciaAux(pos($c_1$, s), pos($c_2$, s))}	

\tadAxioma{loRodeanBien?(cc, n, r, s)}{cantEnemigosCerca(cc, r, s) $\geq$ n}   

\tadAxioma{losEstudiantes(s)}{chabonesConRol(gente(s), estudiante, s)}    

\tadAxioma{losHippies(s)}{chabonesConRol(gente(s), hippie, s)}

\tadAxioma{lospolic\'ias(s)}{chabonesConRol(gente(s), polic\'ia, s)}         

\tadAxioma{posDeCaptura?(cor, s)}{posInmovil?(cor, s) $\wedge$ \\ loRodeanBien?(chabonesEnPosiciones(adyacentesConGente(cor, s), s), 1, polic\'ia, s)}         

\tadAxioma{posDeCapturas(ccor, s)}{\IF $\emptyset$?(ccor) THEN 
				                                $\emptyset$ 
				                            ELSE
				                                {\IF posDeCaptura?(dameUno(ccor), s) THEN
				                                    Ag(dameUno(ccor), posDeCapturas(sinUno(ccor), s))
				                                ELSE
				                                    posDeCapturas(sinUno(ccor), s)
				                                FI}
				                            FI}                               
                            
\tadAxioma{posEstudiantes(s)}{posOcupadasPorGenteAux(losEstudiantes(s), s)}

\tadAxioma{posHippies(s)}{posOcupadasPorGenteAux(losHippies(s), s)}

\tadAxioma{pospolic\'ias(s)}{posOcupadasPorGenteAux(losPolic\'ias(s), s)}

\tadAxioma{posQueSeConvierte?(cor, r, s)}{\IF r = hippie THEN
					                               posInmovil?(cor, s) $\wedge$ \#adyacentesOcupadas(cor, s) = \#adyacentesConGente(cor, s) $\wedge$ \\
					                                loRodeanBien?(chabonesEnPosiciones(adyacentesConGente \\
					                                (cor, s), s), \#adyacentesConGente(cor, s), estudiante, s)
					                         ELSE 
					                                 loRodeanBien?(chabonesEnPosiciones(adyacentesConGente \\
					                                (cor, s), s), 2, hippie, s)
					                          FI}
                            
\tadAxioma{posQueSeConvierten(ccor, d, s)}{\IF $\emptyset$?(ccor) THEN
					                                    $\emptyset$
					                                ELSE
					                                    {\IF posQueSeConvierte?(dameUno(ccor), obtener(dameUno(ccor), d), s) THEN
					                                        Ag(dameUno(ccor), posQueSeConvierten(sinUno(ccor), d, s))
					                                    ELSE
					                                        posQueSeConvierten(sinUno(ccor), d, s)
					                                    FI}
					                                FI}   

\tadAxioma{posInmovil?(cor, s)}{$\#$adyacentesLibres(cor, s) = 0}    

\tadAxioma{posInmoviles(ccor, s)}{\IF $\emptyset$?(ccor) THEN
								$\emptyset$
							ELSE
								{\IF posInmovil?(dameUno(ccor), s) THEN
									Ag(dameUno(ccor), posInmoviles(sinUno(ccor), s))
								ELSE
									posInmoviles(sinUno(ccor), s)
								FI}
							FI}

\tadAxioma{posLibresSist(s)}{posLibres(laGrilla(s)) - posOcupadasPorGente(s)}

\tadAxioma{posOcupadasSist(s)}{posOcupadas(laGrilla(s)) $\cup$ posOcupadasPorGente(s)}

\tadAxioma{posOcupadasPorGente(s)}{posOcupadasPorGenteAux(gente(s), s)}   

\tadAxioma{recordCapturas(s)}{recordCapturasAux(lospolic\'ias(s), 0, s)}   

\tadAxioma{losMasVigilantes(cc, n, s)}{\IF $\emptyset$?(cc) THEN
									$\emptyset$
								ELSE
									{\IF capturasDe(dameUno(cc), s) = n THEN
										Ag(dameUno(cc), losMasVigilantes(sinUno(cc), n, s))
									ELSE
										losMasVigilantes(sinUno(cc), n, s)
									FI}
								FI}	
                                    
\tadAxioma{posInicialAux(c, cc, ccor)}{\IF c = dameUno(cc) THEN
									dameUno(ccor)
								ELSE
									posInicialAux(c, sinUno(cc), sinUno(ccor))
								FI}	
                                    
\tadAxioma{posOcupadasPorGenteAux(cc, s)}{\IF $\emptyset$?(cc) THEN 
                                            $\emptyset$ 
                                          ELSE 
                                            Ag(pos(dameUno(cc), s), posOcupadasPorGenteAux(sinUno(cc), s)) 
                                          FI}
                                          
                                         
                                          
\tadAxioma{chabonEnPosicionAux(cc, cor, s)}{\IF pos(dameUno(cc), s) = cor THEN 
                                                dameUno(cc)
                                            ELSE
                                                chabonEnPosicionAux(sinUno(cc), cor, s)
                                            FI}      
                                    
\tadAxioma{distanciaAux($cor_1$, $cor_2$)}{\IF $cor_1$.x $\geq$ $cor_2$.x THEN 
											{\IF $cor_1$.y $\geq$ $cor_2$.y THEN
												$cor_1$.x - $cor_2$.x + $cor_1$.y - $cor_2$.y 
											ELSE
												distancia(($cor_1$.x, $cor_2$.y), ($cor_2$.x, $cor_1$.y))
											FI}
										ELSE
											distancia(($cor_2$.x, $cor_1$.y), ($cor_1$.x, $cor_2$.y))
										FI}        

\tadAxioma{recordCapturasAux(cc, n, s)}{\IF $\emptyset$?(cc) THEN
									n
								ELSE
									{\IF capturasDe(dameUno(cc), s) $>$ n THEN
										recordCapturasAux(sinUno(cc), capturasDe(dameUno(cc), s), s)
									ELSE
										recordCapturasAux(sinUno(cc), n, s)
									FI}
								FI}
									


\tadAxioma{distUnTargetMasCercano(c, ccor, n, s)}{\IF $\emptyset$?(ccor) THEN
										n
									ELSE
										{\IF distancia(pos(c, s), dameUno(ccor)) < n THEN
											unTargetMasCercano(pos(c, s), sinUno(ccor), distancia(pos(c, s), dameUno(ccor)))
										ELSE
											unTargetMasCercano(pos(c, s), sinUno(ccor), n)
										FI}
									FI}
									
\tadAxioma{conjLosTargetsMasCercanos(c, ccor, n, s)}{\IF $\emptyset$?(cc) THEN
											$\emptyset$
										ELSE
											{\IF distancia(pos(c, s), dameUno(ccor)) = n THEN
												Ag(dameUno(ccor), conjLosTargetsMasCercanos(c, sinUno(ccor), n, s))
											ELSE
												conjLosTargetsMasCercanos(c, sinUno(ccor), n, s)
											FI}
										FI}						                                										                                                                                                                                                                                                             
					
        

\end{tad}


\end{document}
