\documentclass[10pt, a4paper]{article}
% especifico m�rgenes manualmente
\usepackage[paper=a4paper, left=1.5cm, right=1.5cm, bottom=1.5cm, top=3.5cm]{geometry}
% codificaci�n ISO-8859-1
\usepackage[latin1]{inputenc}
% separaci�n sil�bica en castellano
\usepackage[spanish]{babel}
% paquetes y caratula de algo2
\usepackage{aed2-symb,aed2-itef,aed2-tad,caratula}

\begin{document}

% Estos comandos deben ir antes del \maketitle
\materia{Algoritmos y Estructuras de Datos II} % obligatorio
\submateria{Primer Cuatrimestre de 2015} % opcional
\titulo{Trabajo Pr�ctico 1} % obligatorio
\subtitulo{Especificaci�n DCNet} % opcional
\grupo{Grupo 24} % opcional 


\integrante{Fernando Frassia}{340/13}{ferfrassia@gmail.com} % obligatorio 
\integrante{Rodrigo Seoane Quilne}{910/11}{seoane.raq@gmail.com}
\integrante{Sebastian Matias Giambastiani}{916/12}{sebastian.giambastiani@hotmail.com}
\integrante{}{}{}



\maketitle

% compilar 2 veces para actualizar las referencias
\tableofcontents

\pagebreak
\newpage

\section{Aclaraciones}
$\bullet$ \tadNombre {TDC} son todos los caminos completos posibles de una computadora a otra.

$\bullet$ \tadNombre{caminoParcial} genera una secuencia de tuplas (c1, i1, i2, c2). En el �ltimo elemento de la secuencia, se encuentra la informaci�n de d�nde est� el paquete actualmente (c1), cu�l es la pr�xima computadora a la que ir� (c2), y cuales son las interfaces que las conectan (i1 e i2 respectivamente). Notar que caminoParcial no distingue entre un paquete que lleg� a su destino y un paquete que est� en la �ltima computadora antes de llegar a su destino. Esto es por un tema de tipos (dado que la �ltima computadora no puede ser agregada como una tupla (c, i, i, c), por eso es que existe $\pi$$_3$ en el observador infoP, el cual devuelve true cuando p est� en su destino, y false cuando no. Tambi�n vale aclarar que en $\pi$$_1$ se encuentra su computadora origen (es decir, donde se encol�), y en $\pi$$_2$ su computadora destino.

$\bullet$ Se decidi� que las computadoras no puedan estar conectadas consigo mismas

$\bullet$ Si bien el observador caminoParcial devuelve el camino m�nimo de un paquete (a medida que este viaja), se decidi� que el camino m�nimo a seguir (caminoTotal) se genere en la instancia en que se encola dicho paquete. 

$\bullet$ \tadNombre{metoInterfaces} toma una Secu(Compu) y devuelve una Secu(Compu, Interfaz, Interfaz, Compu). No logramos que entrara la signatura en la hoja. Hay un peque�o abuso de notaci�n al no escribir la palabra \tadNombre{tupla} pero es porque as� iba a entrar menos en la hoja y por ende entenderse menos.

$\bullet$ \tadNombre{encolar} no representa una cola ni trabaja con el TAD \tadNombre {cola}, s� hay un comportamiento tipo 'cola' en su axiomatizaci�n. La raz�n de su nombre es porque es como que 'encola' paquetes en una computadora. Pero no hicimos una Cola y tampoco nos interes� observar, desde los observadores, quienes son los paquetes en espera (s� precisamos esta informaci�n en un momento y la obtuvimos mediante otra operaci�n). Solo nos interes� saber cu�ntos hab�a en espera en cada computadora. 
 
 $\bullet$ En la restricci�n de \tadNombre{encolar}, la operaci�n hayConexi�n?(laRed(d), $c_1$, $c_2$) ya chequea que $c_1$ y $c_2$ sean distintas. Por eso no est� explicito en la restricci�n.
 
 $\bullet$ \tadNombre{paquetesDe} y \tadNombre{paquetesDe2} devuelven el conjunto de paquetes 'en transito' de una computadora, es decir, no toman en cuenta los que s� est�n en esa computadora pero ya llegaron a su destino.
 

%RenombresBegin
\section{Renombres}

\tadNombre{Interfaz} es \tadNombre{Nat}

\tadNombre{Compu} es \tadNombre{Nat}

\tadNombre{Paquete} es \tadNombre{Tupla(Nat, Nat), el $\pi$$_1$ de la tupla es la prioridad del paquete, y el $\pi$$_2$ es su identificador personal}

%\tadNombre{} es \tadNombre{}
%RenombresEnd

%TADSBegin

%SecuExtendidoBegin
\section{\tadNombre {Operaciones Auxiliares}}

%\begin{tad}{\tadNombre{}}
%\tadIgualdadObservacional{}{}{}{}
%\tadParametrosFormales{\tadGeneros{$\alpha$}}
%\tadGeneros{Secu($\alpha$)}

%\tadExporta{Secu, generadores, observadores}
%\tadUsa{}

%\tadObservadores
%\tadGeneradores
\tadOtrasOperaciones
\tadAlinearFunciones{agATodos}{Secu($\alpha$)/s1, Secu($\alpha$)/s2}

\tadOperacion{aConjunto}{Secu($\alpha$)}{Conj($\alpha$)}{}

\tadOperacion{siguiente}{Secu($\alpha$)/s1, Secu($\alpha$)/s2}{$\alpha$}{long($s_1$) $\textless$ long($s_2$)}

\tadOperacion{masCorto}{Conj(Secu($\alpha$))/cs}{Secu($\alpha$)}{$\neg$$\emptyset$?(cs)}

\tadOperacion{agATodos}{$\alpha$, Conj(Secu($\alpha$))}{Conj(Secu($\alpha$))}{}

\tadAxiomas[\paratodo{$\alpha$}{e}, \paratodo{Conj($\alpha$)}{c}, \paratodo{Secu($\alpha$)}{s, s1, s2}. \paratodo{Conj(Secu($\alpha$))}{cs}]
\tadAlinearAxiomas{masChica?(s1, s2)}

\tadAxioma{aConjunto(s)}{\IF vacia?(s) THEN $\emptyset$ ELSE Ag(prim(s), aConjunto(fin(s))) FI}

\tadAxioma{siguiente($s_1$, $s_2$)}{\IF vacia?($s_1$) THEN prim($s_2$) ELSE siguiente(fin($s_1$), fin($s_2$)) FI}

\tadAxioma{masCorto(cs)}{\IF $\#$(cs) = 1 THEN dameUno(cs) ELSE {\IF long(dameUno(cs)) $\textless$ long(dameUno(SinUno(cs))) THEN masCorto(cs - dameUno(sinUno(cs))) ELSE masCorto(sinUno(cs)) FI} FI}

\tadAxioma{agATodos(e, cs)}{\IF $\emptyset$?(cs) THEN $\emptyset$ ELSE Ag((e $\bullet$ dameUno(cs)), agATodos(e, sinUno(cs))) FI}


%\end{tad}
%SecuExtendidoEnd
\newpage

%RedBegin
\section{TAD \tadNombre{Red}}

\begin{tad}{\tadNombre{Red}}
\tadIgualdadObservacional{r1}{r2}{Red}{compus($r_1$) \igobs compus($r_2$) $\yluego$ \\ ($\forall$ c: Compu) \\ $[$(c $\in$ compus($r_1$) $\impluego$ (interfacesDe($r_1$, c) \igobs interfacesDe($r_2$, c))) $\yluego$ \\ ($\forall$ i:Interfaz) \\ $[$(c $\in$ compus($r_1$) $\yluego$ i $\in$ interfacesDe($r_1$, c) $\impluego$ (iLibre?($r_1$, c, i) \igobs iLibre?($r_2$, c, i))) $\yluego$ \\ (c $\in$ compus($r_1$) $\yluego$ i $\in$ interfacesDe($r_1$, c) $\yluego$ $\neg$iLibre?($r_1$, c, i) $\impluego$ (conectadoA($r_1$, c, i) \igobs conectadoA($r_2$, c, i)))$]$$]$}
\tadGeneros{Red}


\tadExporta{Red, generadores, observadores}
\tadUsa{\tadNombre{Bool}, \tadNombre{Conj($\alpha$)}, \tadNombre{Compu}, \tadNombre{Interfaz}, \tadNombre{Secu($\alpha$)}}

\tadObservadores
\tadAlinearFunciones{interfacesDe}{Red/r, Compu/c, Interfaz/i}
\tadOperacion{compus}{Red}{Conj(Compu)}{}
\tadOperacion{interfacesDe}{Red/r, Compu/c}{Conj(Interfaz)}{c $\in$ compus(r)}
\tadOperacion{conectadoA}{Red/r, Compu/c, Interfaz/i}{Compu}{c $\in$ compus(r) $\yluego$ i $\in$ interfacesDe(r,c) $\yluego$ $\neg$ iLibre?(r, c, i)}
\tadOperacion{iLibre?}{Red/r, Compu/c, Interfaz/i}{Bool}{c $\in$ compus(r) $\yluego$ i $\in$ interfacesDe(r, c)}

\tadGeneradores
\tadAlinearFunciones{conectar}{Red/r, Compu/c1, Compu/c2, Interfaz/i1, Interfaz/i2}
\tadOperacion{crear}{}{Red}{}
\tadOperacion{agCompu}{Red/r, Compu/c, Conj(Interfaz)}{Red}{c $\notin$ compus(r)}
\tadOperacion{conectar}{Red/r, Compu/c1, Compu/c2, Interfaz/i1, Interfaz/i2}{Red}{$c_1$ $\neq$ $c_2$ $\wedge$ $\{$$c_1$, $c_2$$\}$ $\subseteq$ compus(r) $\yluego$ $c_1$ $\notin$ compusDirectas(r,$c_2$) $\wedge$ $c_2$ $\notin$ compusDirectas(r,$c_1$) $\wedge$ (($i_1$ $\in$ interfacesDe(r,$c_1$) $\wedge$ $i_2$ $\in$ interfacesDe(r,$c_2$)) $\yluego$ iLibre?(r,$c_1$,$i_1$) $\wedge$ iLibre?(r,$c_2$,$i_2$))} 

\tadOtrasOperaciones
\tadAlinearFunciones{conjHayConexion?}{Red/r, Conj(Compu)/cc1, Compu/c, Conj(Compu)/cc2}

\tadOperacion{iOcupadas}{Red/r, Compu/c}{Conj(Interfaz)}{c $\in$ compus(r)}

\tadOperacion{iOcupadas2}{Red/r, Compu/c, Conj(Interfaz)/ci}{Conj(Interfaz)}{c $\in$ compus(r) $\yluego$ $c_i$ $\subseteq$ interfacesDe(r,c)}

\tadOperacion{hayConexion?}{Red/r, Compu/c1, Compu/c2}{Bool}{$\{$$c_1$, $c_2$$\}$ $\subseteq$ compus(r)}

\tadOperacion{hayConexion2?}{Red/r, Compu/c1, Compu/c2, Conj(Compu)/cc}{Bool}{($\{$$c_1$, $c_2$$\}$ $\cup$ cc) $\subseteq$ compus(r)}

\tadOperacion{conjHayConexion?}{Red/r, Conj(Compu)/cc1, Compu/c, Conj(Compu)/cc2}{Bool}{($\{$c$\}$ $\cup$ cc1 $\cup$ cc2) $\subseteq$ compus(r)}

\tadOperacion{compusDirectas}{Red/r, Compu/c}{Conj(Compu)}{c $\in$ compus(r)}

\tadOperacion{compusDirectas2}{Red/r, Compu/c, Conj(Interfaz)/ci}{Conj(Compu)}{c $\in$ compus(r) $\yluego$ ci $\subseteq$ iOcupadas(r, c)}

\tadOperacion{TDC}{Red/r, Compu/c1, Compu/c2}{Conj(Secu(Compu))}{$\{$$c_1$, $c_2$$\}$ $\subseteq$ compus(r)}

\tadOperacion{TDC2}{Red/r, Compu/c1, Compu/c2, Secu(Compu)/sc}{Conj(Secu(Compu))}{$\{$$c_1$, $c_2$$\}$ $\subseteq$ compus(r)}

\tadOperacion{conjTDC}{Red/r, Conj(Compu)/cc, Compu/c, Secu(Compu)/sc}{Conj(Secu(Compu))}{Ag(c, cc) $\subseteq$ compus(r)}

\tadOperacion{suInterfaz}{Red/r, Compu/c1, Compu/c2}{Interfaz}{$\{$$c_1$, $c_2$$\}$ $\subseteq$ compus(r) $\yluego$ $c_2$ $\in$ compusDirectas(r, $c_1$)}

\tadOperacion{suInterfaz2}{Red/r, Compu/c1, Compu/c2, Conj(Interfaz)/ci}{Interfaz}{$\{$$c_1$, $c_2$$\}$ $\subseteq$ compus(r) $\yluego$ ci $\subseteq$ iOcupadas(r, $c_1$)}

\tadOperacion{metoInterfaces}{Red/r, Secu(Compu)/sc}{Secu(Compu, Interfaz, Interfaz, Compu)}{$\neg$vacia?(sc)}
  

\tadAxiomas[\paratodo{Red}{r}, \paratodo{Compu}{c, c1, c2, c3}, \paratodo{Interfaz}{i, i1, i2, i3}, \paratodo{Conj(Interfaz)}{ci}, \paratodo{Conj(Compu)}{cc, cc1, cc2}, \paratodo{Secu(Compu)}{sc}]
\tadAlinearAxiomas{conectadoA(conectar(r, c1, c2, i1, i2), c3, i3)}

\tadAxioma{compus(crear())}{$\emptyset$}
\tadAxioma{compus(AgCompu(r, c, ci))}{Ag(c , compus(r))}
\tadAxioma{compus(conectar(r, $c_1$, $c_2$, $i_1$, $i_2$))}{compus(r)}

\tadAxioma{interfacesDe(agCompu(r, $c_1$, ci), $c_2$)}{\IF $c_1$ = $c_2$ THEN ci ELSE interfacesDe(r, $c_2$) FI}
\tadAxioma{interfacesDe(conectar(r, $c_1$, $c_2$, $i_1$, $i_2$), $c_3$)}{interfacesDe(r, $c_3$)}

\tadAxioma{conectadoA(agCompu(r, $c_1$, ci), $c_2$, $i_2$)}{conectadoA(r, $c_2$, $i_2$)}
\tadAxioma{conectadoA(conectar(r, $c_1$, $c_2$, $i_1$, $i_2$), $c_3$, $i_3$)}{\IF $c_3$ = $c_1$ $\wedge$ $i_3$ = $i_1$ THEN $c_2$ ELSE {\IF $c_3$ = $c_2$ $\wedge$ $i_3$ = $i_2$ THEN $c_1$ ELSE conectadoA(r, $c_3$, $i_3$) FI} FI}

\tadAxioma{iLibre?(agCompu(r, $c_1$, ci), $c_2$, i)}{\IF $c_1$ = $c_2$ THEN true ELSE iLibre?(r, $c_2$, i) FI}
\tadAxioma{iLibre?(conectar(r, $c_1$, $c_2$, $i_1$, $i_2$), $c_3$, $i_3$)}{\IF $c_3$ = $c_1$ $\wedge$ $i_3$ = $i_1$ THEN false ELSE {\IF $c_3$ = $c_2$ $\wedge$ $i_3$ = $i_2$ THEN false ELSE iLibre?(r, $c_3$, $i_3$) FI} FI}

\tadAxioma{iOcupadas(r, c)}{iOcupadas2(r, c, interfacesDe(r, c))}

\tadAxioma{iOcupadas2(r, c, ci)}{\IF $\emptyset$?(ci) THEN $\emptyset$ ELSE {\IF $\neg$ iLibre?(r, c, dameUno(ci)) THEN Ag(dameUno(ci), iOcupadas2(r, c, sinUno(ci))) ELSE iOcupadas2(r, c, sinUno(ci)) FI} FI}

\tadAxioma{hayConexion?(r, $c_1$, $c_2$)}{hayConexion2?(r, $c_1$, $c_2$, $\emptyset$)}

\tadAxioma{hayConexion2?(r, $c_1$, $c_2$, cc)}{\IF $c_2$ $\in$ compusDirectas(r, $c_1$) THEN true ELSE conjHayConexion?(r, compusDirectas(r, $c_1$) - cc, $c_2$, Ag($c_1$, cc)) FI} 

\tadAxioma{conjHayConexion?(r, $cc_1$, c, $cc_2$)}{\IF $\emptyset$?($cc_1$) THEN false ELSE hayConexion2?(r, dameUno($cc_1$), c, $cc_2$) $\vee$ conjHayConexion?(r, sinUno($cc_1$), c, $cc_2$) FI}

\tadAxioma{compusDirectas(r, c)}{compusDirectas2(r, c, iOcupadas(r, c))}

\tadAxioma{compusDirectas2(r, c, ci)}{\IF $\emptyset$?(ci) THEN $\emptyset$ ELSE Ag((conectadoA(r, c, dameUno(ci))), compusDirectas2(r, c, sinUno(ci))) FI}

\tadAxioma{TDC(r, $c_1$, $c_2$)}{agATodos($c_1$, TDC2(r, $c_1$, $c_2$, $\textless$$\textgreater$))}

\tadAxioma{TDC2(r, $c_1$, $c_2$, sc)}{\IF $c_2$ $\in$ compusDirectas(r, $c_1$) THEN $\{$sc $\&$ $\textless$$c_1$, $c_2$$\textgreater$$\}$ ELSE conjTDC(r, compusDirectas(r, $c_1$) - aConjunto(sc), $c_2$, sc  $\circ$ $c_1$) FI}

\tadAxioma{conjTDC(r, cc, c, sc)}{\IF $\emptyset$?(cc) THEN $\emptyset$ ELSE TDC2(r, dameUno(cc), c, sc) $\bigcup$ conjTDC(r, sinUno(cc), c, sc) FI}

\tadAxioma{suInterfaz(r, $c_1$, $c_2$)}{suInterfaz2(r, $c_1$, $c_2$, iOcupadas(r, $c_1$))}

\tadAxioma{suInterfaz2(r, $c_1$, $c_2$, ci)}{\IF conectadoA(r, $c_1$, dameUno(ci)) = $c_2$ THEN dameUno(ci) ELSE suInterfaz2(r, $c_1$, $c_2$, sinUno(ci)) FI}

\tadAxioma{metoInterfaces(r, sc)}{\IF vacia?(fin(sc)) THEN $\textless$$\textgreater$ ELSE (prim(sc), suInterfaz(r, prim(sc), prim(fin(sc))), suInterfaz(r, prim(fin(sc)), prim(sc)), prim(fin(sc))) $\bullet$ metoInterfaces(r, fin(sc)) FI}


\end{tad}
%RedEnd

\newpage

%DCNetBegin
\section{TAD \tadNombre{DCNet}}

\begin{tad}{\tadNombre{DCNet}}
\tadIgualdadObservacional{d1}{d2}{DCNet}{laRed($d_1$) $\igobs$ laRed($d_2$)  $\wedge$ \\ paquetes($d_1$) $\igobs$ paquetes($d_2$) $\yluego$ \\ $[$($\forall$c: Compu) (c $\in$ compus(laRed($d_1$)) $\impluego$ (cantEnviados($d_1$, c) $\igobs$ cantEnviados($d_2$, c))) $\wedge$ \\ ($\forall$p: Paquete) (p $\in$ paquetes($d_1$) $\impluego$ (infoP($d_1$, p) $\igobs$ infoP($d_2$, p) $\wedge$ caminoParcial($d_1$, p) $\igobs$ caminoParcial($d_2$, p)))$]$}
\tadGeneros{DCNet}

\tadExporta{DCNet, generadores, observadores}
\tadUsa{\tadNombre{Red}, \tadNombre{Secu($\alpha$)}, \tadNombre{Conjunto($\alpha$)}, \tadNombre{Nat}, \tadNombre{Bool}, \tadNombre{Paquete}, \tadNombre{Tupla}}

\tadObservadores
\tadAlinearFunciones{cantEnviados}{DCNet/d, Compu/c}
\tadOperacion{laRed}{DCNet}{Red}{}
\tadOperacion{paquetes}{DCNet}{Conj(Paquete)}{}
\tadOperacion{infoP}{DCNet/d, Paquete/p}{(Compu, Compu, Bool)}{p $\in$ paquetes(d)}
\tadOperacion{cantEnviados}{DCNet/d, Compu/c}{Nat}{c $\in$ compus(laRed(d))}
\tadOperacion{caminoParcial}{DCNet/d, Paquete/p}{Secu(Compu, Interfaz, Interfaz, Compu)}{p $\in$ paquetes(d)}

\tadGeneradores
\tadAlinearFunciones{pasarSeg}{DCNet/d, Compu/c1, Compu/c2, Paquete/p}
\tadOperacion{iniciar}{Red}{DCNet}{}
\tadOperacion{encolar}{DCNet/d, Compu/c1, Compu/c2, Paquete/p}{DCNet}{$[$($\forall$ $p_2$: Paquete) ($p_2$ $\in$ paquetes(d) $\Rightarrow$ $\pi$$_2$($p_2$) $\neq$ $\pi$$_2$(p))$]$  $\wedge$ $\{$$c_1$, $c_2$$\}$ $\subseteq$ compus(laRed(d)) $\yluego$ hayConexion?(laRed(d), $c_1$, $c_2$)}
\tadOperacion{pasarSeg}{DCNet}{DCNet}{}



\tadOtrasOperaciones
\tadAlinearFunciones{conjMasEnviante}{DCNet/d, Compu/c1, Compu/c2, Paquete/p}
\tadOperacion{masEnviante}{DCNet/d}{Compu}{$\neg$$\emptyset$?(compus(laRed(d)))}
\tadOperacion{masEnviante2}{DCNet/d, Conj(Compu)/cc}{Compu}{cc $\subseteq$ compus(laRed(d)) $\wedge$ $\neg$ $\emptyset$?(cc)}
\tadOperacion{conjMasEnviante}{DCNet/d, Conj(Compu)/cc, Compu/c}{Conj(Compu)}{Ag(c, cc) $\subseteq$ compus(laRed(d))}
\tadOperacion{caminoTotal}{DCNet/d, Paquete/p}{Secu(Compu, Interfaz, Interfaz, Compu)}{p $\in$ paquetes(d)}
\tadOperacion{paquetesDe}{DCNet, Compu}{Conj(Paquete)}{}
\tadOperacion{paquetesDe2}{DCNet/d, Compu, Conj(Paquete)/cp}{Conj(Paquete)}{cp $\subseteq$ paquetes(d)}

\tadOperacion{masPrioritario}{Conj(Paquete)/cp}{Paquete}{$\neg$$\emptyset$?(cp)}


\tadAxiomas[\paratodo{Red}{r}, \paratodo{DCNet}{d}, \paratodo{Compu}{c, c1, c2, c3}, \paratodo{Paquete}{p}, \paratodo{Conj(Compu)}{cc}, \paratodo{Conj(Paquete)}{cp}]
\tadAlinearAxiomas{caminoParcial(encolar(d, c1, c2, p1), p2)}
\tadAxioma{laRed(iniciar(r))}{r}
\tadAxioma{laRed(encolar(d, $c_1$, $c_2$, p))}{laRed(d)}
\tadAxioma{laRed(pasarSeg(d))}{laRed(d)}

\tadAxioma{paquetes(iniciar(r))}{$\emptyset$}
\tadAxioma{paquetes(encolar(d, $c_1$, $c_2$, p))}{Ag(p, paquetes(d))}
\tadAxioma{paquetes(pasarSeg(d))}{paquetes(d)}

\tadAxioma{infoP(encolar(d, $c_1$, $c_2$, $p_1$), $p_2$)}{\IF $p_1$ = $p_2$ THEN ($c_1$, $c_2$, false) ELSE infoP(d, $p_2$) FI}
\tadAxioma{infoP(pasarSeg(d), p)}{\IF $\pi$$_4$(ult(caminoParcial(d, p))) = $\pi$$_2$(infoP(d, p)) $\wedge$ \\ $\neg$$\pi$$_3$(infoP(d, p)) $\yluego$ \\ masPrioritario(paquetesDe(d, $\pi$$_1$(ult(caminoParcial(d, p))))) = p\\ THEN ($\pi$$_1$(infoP(d, p)), $\pi$$_2$(infoP(d, p)), true) ELSE infoP(d, p) FI}

\tadAxioma{cantEnviados(iniciar(r), c)}{0}
\tadAxioma{cantEnviados(encolar(d, $c_1$, $c_2$, p), $c_3$)}{cantEnviados(d, $c_3$)}
\tadAxioma{cantEnviados(pasarSeg(d), c)}{\IF $\#$paquetesDe(d, c) = 0 THEN cantEnviados(d, c) ELSE cantEnviados(d, c) + 1 FI}

\tadAxioma{caminoParcial(encolar(d, $c_1$, $c_2$, $p_1$), $p_2$)}{\IF $p_1$ = $p_2$ THEN $\textless$prim(caminoTotal(encolar(d, $c_1$, $c_2$, $p_1$), $p_2$))$\textgreater$ ELSE caminoParcial(d, $p_2$) FI}

\tadAxioma{caminoParcial(pasarSeg(d), p)}{\IF $\pi$$_4$(ult(caminoParcial(d, p))) = $\pi$$_2$(infoP(d, p)) THEN caminoParcial(d, p) ELSE {\IF masPrioritario(paquetesDe(d, $\pi$$_1$(ult(caminoParcial(d, p))))) = p THEN caminoParcial(d, p) $\circ$ siguiente(caminoParcial(d, p), caminoTotal(d, p))  ELSE caminoParcial(d, p) FI} FI}

\tadAxioma{masEnviante(d)}{dameUno(conjMasEnviante(d, compus(laRed(d)), masEnviante2(d, compus(laRed(d)))))}

\tadAxioma{masEnviante2(d, cc)}{\IF \#(cc) = 1 THEN dameUno(cc) ELSE {\IF cantEnviados(d, dameUno(cc)) $\geq$ cantEnviados(d, dameUno(sinUno(cc))) THEN masEnviante2(d, cc - dameUno(sinUno(cc))) ELSE masEnviante2(d, cc - dameUno(cc)) FI} FI}

\tadAxioma{conjMasEnviante(d, cc, c)}{\IF $\emptyset$?(cc) THEN  $\emptyset$ ELSE {\IF cantEnviados(d, dameUno(cc)) = cantEnviados(d, c) THEN Ag(dameUno(cc), conjMasEnviante(d, sinUno(cc), c)) ELSE conjMasEnviante(d, sinUno(cc), c) FI} FI}

\tadAxioma{caminoTotal(d, p)}{metoInterfaces(laRed(d), masCorto(TDC(laRed(d), $\pi$$_1$(infoP(d, p)), $\pi$$_2$(infoP(d, p)))))}

\tadAxioma{paquetesDe(d, c)}{paquetesDe2(d, c, paquetes(d))}

\tadAxioma{paquetesDe2(d, c, cp)}{\IF $\emptyset$?(cp) THEN $\emptyset$ ELSE {\IF $\pi$$_1$(ult(caminoParcial(d, dameUno(cp)))) = c $\wedge$ \\ $\neg$$\pi$$_3$(infoP(d, dameUno(cp))) \\ THEN Ag(dameUno(cp), paquetesDe2(d, c, sinUno(cp))) ELSE paquetesDe2(d, c, sinUno(cp)) FI} FI}

\tadAxioma{masPrioritario(cp)}{\IF $\#$(cp) = 1 THEN dameUno(cp) ELSE {\IF $\pi$$_1$(dameUno(cp)) $\textgreater$ $\pi$$_1$(dameUno(sinUno(cp))) THEN masPrioritario(cp - dameUno(sinUno(cp))) ELSE masPrioritario(cp - dameUno(cp)) FI} FI}




\end{tad}
%DCNetEnd
%TADSEnd

\end{document}
